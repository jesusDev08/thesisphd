\chapter{Classifiers}
In this chapter we provide a background of classifiers in ISL. We first provide an 
account of classifiers in spoken language and sign language. We delve into  the 
morphological, grammatical details, handshapes conveyed by classifiers within ISL as 
described in literature. 


\section{What is a classifier?}

Classifiers are commonly found in nearly all the sign languages examined so far. They 
constitute a thoroughly investigated area within the field of sign language 
linguistics. Nevertheless, there is still considerable ongoing discussion regarding 
various aspects of these elements \parencite{zwitserlood2012}.
The beginning of research into classifiers within sign languages coincided with the 
renewed interest in studying classifiers in spoken languages. \parencite{zwitserlood2012}. 

Having defined classifiers as morphemes that indicate a semantic class of items 
belong. Various analyses of classifiers have been presented in the literature on the 
linguistics of spoken language \parencite{morgan2007}. 

To mention a remarkable work made by Allan (1977). He proposed four distinct 
classifiers languages. Sign language classifiers align with one of these classifiers 
called predicate classifiers. To demonstrate this, they made the comparison of 
structures in Navajo and ASL; this comparison is based on a misinterpretation of 
classificatory verbs of Navajo (Engberg-Pedersen 1993; Schembri 2001; Zwitserlood 
1996, 2003). \textcite{vermeerbergen2023} stated that the previous comparison 
and due to various factors, including the issues discussed extensively by \textcite{schembri2003}, several sign linguists have opposed the interpretation of the handshape 
within what is commonly referred to as "classifier constructions" as classifiers. 
Moreover, the concept that every constituent element of these constructions is 
separate, can be listed, and is explicitly delineated within the grammar of specific 
sign languages, each possessing morphemic status (as proposed by Supalla in 1982), 
has also been subject to scrutiny and scepticism.

According to \textcite{zwitserlood2012}, classifiers are primarily conveyed through specific 
arrangements of the manual articulator and represent entities by signifying their 
prominent and distinctive attributes. 

\textcite{morgan2007} mention three types of verbs (plain, agreeing, and spatial 
verbs) where classifiers fall in the category of spatial verbs. Spatial verb 
classification comprises the verbs of movement and location, often called "classifier 
verbs." So, classifier verbs describe essential details regarding the path, 
trajectory, speed, and spatial location of the depicted action or movement. Then, 
classifier predicates are predicates of direction or location \parencite{zwitserlood2012}.

In Australian sign language (Auslan),\textcite{schembri1996structure} shows that Auslan has similar 
constructions identified in other sign languages. This category was called 
polycomponential verbs (PVs) of motion, location, and visual-geometric description.
Schembri points out that the significant element represented by handshape in PVs 
has commonly been defined as a classifier morpheme. This is due to the apparent 
variation in handshape selection based on the prominent attributes of the referent, 
particularly its shape.

\textcite{valli2001linguistics} defined a classifier in ASL as a handshape combined 
with location, orientation, movement, and nonmanual signals to form a predicate. For 
their part \textcite{valli2001linguistics}, use the English sentence \textit{The car drove past 
}as an example of a classifier. Here, the sign car is used, followed by a sign with a 
3 handshape, moving from right to left in front of the signer, with the palm facing 
in. A sign with the same handshape can be used to talk about the movement of a boat or a bicycle.
\newpage

\subsection{Classifier types}

In terms of categorisation, Zwitzerlood 2012 analyses the classification provided 
by Supalla (1982, 1986), which divided the ASL classifiers into five types: 

\begin{table}[H]
    \centering  
\begin{tabular}{ |p{4cm}|p{7cm}|}
\hline
\multicolumn{2}{|c|}{Categorization of classifiers} \\
\hline
 Classifiers & Description \\
\hline
Semantic Classifiers & Represent nouns by some semantic characteristic of their 
referents (e.g., belonging to the class of humans, animals, or vehicles) \\ \hline
Size and Shape Specifiers (SASSes) & Denote nouns according to the visual geometric 
features of their referents. 
SASSes come in two subtypes:
\begin{itemize}
\item Static SASSes: Consist of a handshape (or combination of two hands) that 
indicates the size/shape of an entity.
\item Tracing SASSes: These have a movement of the hand(s) that outlines an entity’s 
size/shape and in which the shape of the manual articulator denotes the 
dimensionality of that entity.
\end{itemize}
\\ \hline
Instrumental Classifiers & Instrumental comes in two types:
 Hand classifiers, in which the hand represents a hand that holds and/or manipulates 
 another entity.
 Tool classifiers, in which the hand represents a tool that is being manipulated.\\ 
 \hline
Bodypart Classifiers   & Body parts represent themselves (e.g., hands, eyes) or limbs 
(e.g., hands, feet). \\ \hline
Body Classifier & The body of the signer represents an animate entity.  \\
\hline
\end{tabular}
\caption{ASL Classifiers}
\label{tab:asl_classifiers}
\end{table}


Schembri (1996, 1998) categorize PVs into three subcategories: verbs of motion and location, handling, and visual-geometric description. Also, \textcite{zwitserlood2012} mentioned that most researchers no longer view the body classifier as a classifier; instead, they see it as a mechanism for shifting references (Engberg-Pedersen, 1995; Morgan and Woll, 2003). Additionally, \textcite{zwitserlood2012}  argues that  SASSes do not belong to the classifiers domain for the following reasons: 
A mere hand configuration does not express them, and they also need the tracing movement to indicate the shape of the referent.
They cannot be combined with verbs of motion. 
They denote specific shape information (all shapes can be outlined, from square to star-shaped to Italy-shaped). They can be used in various syntactic contexts: they appear as nouns, adjectives, and (ad)verbs and do not seem to be used anaphorically.\\

According to \cite{kimmelman2020}, the following classifier types are well-known:


\begin{table}[H]
\centering 
\begin{tabular}{ |p{4cm}|p{7cm}|}
\hline
\multicolumn{2}{|c|}{\textbf{Classifiers categories}} \\
\hline
 \textbf{Classifiers} & \textbf{Description} \\
\hline
Whole-entity classifiers & Whole-entity classifiers refer to whole objects whose movement or location is described by the predicate. They can be further divided into:
\begin{itemize}
    \item Semantic classifiers refer to a semantic class of objects (humans, trees, cars). 
    \item Size-and-shape specifiers refer to some formal characteristics of the contour of objects (thin objects, round objects, etc.). 
\end{itemize}\\
\hline
Body-part classifiers & Where the hand refers to a body part: a hand, a head, a leg, a tail, etc.\\ 
\hline
Handling classifier & the hand refers to a hand or another manipulator handling some object.\\ 
\hline
\end{tabular}
\caption{Classifiers categories}
\label{tab:asl_classifiers}
\end{table}

The literature researchers recognise two primary classifier categories: Whole Entity 
classifiers and Handling classifiers. Whole Entity classifiers are typically utilised 
in verbs that convey the movement of a referent, its spatial placement, or its 
presence within a given space. In such instances, these classifiers directly depict 
the referent. Conversely, Handling classifiers find their application in verbs that 
portray the manipulated motion or the act of holding a referent \parencite{zwitserlood2012}. 
Also, Zwiterslood mentions a relation between the type of classifier and the verb 
transitivity. In the same way, Benedicto and Brentari (2004) furthermore assert that 
the classifier that is attached to the verb is also responsible for its 
(in)transitivity: a Handling Classifier turns a (basically intransitive) verb into a 
transitive verb.

Regarding the classifier's form, denotation and variation, \textcite{zwitserlood2012} states 
that signers in most sign languages can employ multiple classifiers when representing 
an entity. This allows them to emphasise a specific, distinct characteristic of the 
entity or, conversely, to de-emphasise it.

In terms of classifiers verbs, Supalla (1982) and subsequent studies (Benedicto and 
Brentari, 2004; Chang et al., 2005 Cuxac, 2003; Glück and Pfau, 1998, 1999; 
Zwitserlood, 2003, 2008) proposes the perspective that classifiers function as 
agreement markers or proforms for the referent concerning the verb.

\subsection{Classifiers in ISL}

As mentioned before, predicate classifiers are aligned to signed languages. According 
to that, Brennan (1992) identifies six types of classifier handshapes: semantic, size 
and shape, tracing, instrumental, handling and touch. Based on Brennan's work, 
McDonnell (1996) categorised the predicate classifiers into four categories: Whole 
entity-CL, Extension-CL, Handle entity-CL, and Body-CL. 

\subsubsection{The whole entity-CL stems}

This type of classifier stems from the hand configuration, which usually signifies a 
complete entity. McDonnell (1996) suggests these stems can be paired with the same 
types of movements in ISL, specifically MOVE, BE-LOCATED, and EXIST. Leeson and Saeed 
(2012) identify the following subcategories of whole entity-CLs: Semantic-CL stems, 
which describe entities based on their semantic characteristics (e.g., those that are 
+animate). Size and shape-CL stems represent entities in terms of their shape (e.g., 
rectangular) or their dimensions (e.g., 'two-dimensional object').
McDonnell (ibid) defines the semantic-CL as the ‘multiple entity-CL handshape’. This 
is identifiable as the ‘5-hand/s’; typically, the multiple entity-CL handshapes 
signifies entities as members of large groups.

\subsubsection{Extension-CL stems}

Extension-CL are those stems that trace size and shape the entities they refer to. 
These stems can be paired only with EXTEND movements. 

\subsubsection{Handle Entity-CL Stems}

Handle entity-CL stems denote, with the configuration of the hand, how an actor 
moves, touches or uses an object or part of an object rather than the object as a 
whole. These stems combine the following ISL movements: MOVE, BE-LOCATED and EXIST 
(Leeson and Saeed 2012) \cite{leeson2012}.

\subsubsection{Body-CL stems}

Body-CL stems use the signer's body as an independent articulator to refer to a 
single animate entity. This classifier in ISL refers to the actual body of the 
animate entity rather than the semantic category of the entity's shape. Also, Leeson 
and Saeed (2012) \cite{leeson2012} suggest using these stems when the signer's body functions similarly 
to the way that handshape functions in two-handed configurations.



\subsubsection{Terminology of classifiers}

Schembri (2003), in his terminology analysis of handshape units for describing these 
complex constructions in various sign languages, shows a diversity of terms. For 
instance, in  Australia, they are called classifiers signs or classifiers (Bernal, 
1997 and Branson et al., 1995). Other regions and countries have different names, 
such as classifier verbs or verbs of motion and location (Supalla, 1986, 1990), 
and classifier predicates ( Frishberg, 1975. Kegl and Schley, 1986. Corazza, 1990. Schick, 1987, 1990. Smith, 1990, Valli 
and Lucas, 1995. ), spatially descriptive signs (DeMatteo, 1977), spatial-locative predicates (as mentioned by Liddell and 
Johnson, 1987), polymorphemic predicates (as discussed by Collins-Ahlgren, 1990, and Wallin, 
1990), polysynthetic signs (Takkinen, 1996, and Wallin, 1996, 1998), productive signs 
(Brennan, 1992. Vermeerbergen, 1996 and  Wallin, 1998), polycomponential signs (Slobin et al., 2001), Polymorphemic verbs 
(as described by Engberg-Pedersen, 1993. Schembri, 2003). Depicting verbs (Liddell, 2003; Dudis, 2004) and depicting signs 
(Johnston and Schembri, 2007. Ferrara, 2012).\\

\textcite{schembri2003} chose the term polycomponential verb following the work of Slobin et al. (2000) to refer to 
the classifiers rather than other authors' alternatives. Because of the following 
reasons: First, he stated that the allegation that these forms include classifier 
morphemes is debatable. Second, analyzing these as a multimorphemic is tricky 
(Cogill,1999).
% add text about monomorphemes 
According to \textcite{schembri2003}, This lack of consensus on categorising 
handshape units within PVs may reflect the inherent complexities in linguistic 
descriptions of these forms. Various researchers have proposed markedly different 
analyses regarding delineating the subclasses of handshape units in PVs. \\

\begin{table}[H]
 \centering  
 \begin{tabular}{ |p{2cm}|p{3cm}|p{3cm}|p{3cm}|}
\hline
\multicolumn{4}{|c|}{\textbf{Classification of Classifiers in Signed Languages}} \\
\hline
 \textbf{Author} & \textbf{Entity Handshape Units} & \textbf{Handle Handshape Units} 
 & \textbf{SASS Handshape Units} \\
 \hline
 Supalla & Static SASSes, semantic, bodypart, some instrument classifiers & Some 
 instrument classifiers & Non-static SASSes \\
 \hline
 McDonald & Some x-type of object classifiers & Handle x-type of object & 
 Some x-type of object \\
 \hline
 Shepard-Kegl & Shape/object classifiers & Handling classifiers & - \\
 \hline
 Johnston & Substitutors/proforms & Some manipulators & Some manipulators \\
 \hline
 Corazza  & Surface, some grab, perimeter and some quantity (?) classifiers
 & Some grab classifiers & Descriptive, some perimeter and some quantity (?) 
 classifiers \\ 
 \hline
 Brennan & Semantic classifiers, some SASSes & Handling, instrumental and
touch classifiers & Tracing classifiers and some SASSes \\
\hline
Schick & Class classifiers, some SASSes & Handle classifiers & Some SASSes \\
\hline
Engberg-Pedersen & Whole entity stems, some limb stems & Handle stems and some limb 
stems & Extension stems \\
\hline
Liddell \& Johnson & Whole entity, surface, on-surface classifiers and some
extent (?) classifiers & Instrumental classifiers &  Depth and width,
perimeter-shape and some extent (?) classifiers \\
\hline
Zwitserlood & Object & Handle & - \\
\hline
\end{tabular}
\caption{Classifiers in Signed Languages taken from \parencite{schembri2003}:10}
\label{tab:classifiers_signed_languages}
\end{table}


\section{Depicting verbs}

Liddell pioneered a novel perspective on classifier constructions, reimagining them as depicting signs, distinct from fully lexical signs due to their unique characteristic of "depicting certain aspects of their meanings" (Liddell, 2003:261). Liddell posits that these signs possess a blend of both descriptive and depictive qualities. In his work, Liddell (ibid) demonstrates that signers can employ partial lexical classifier constructions and/or constructed action to craft topographical real-space blends. On the one hand, signers can craft depicting blends, manifesting as small-scale representations of the event space they intend to convey in the sign space right before them \parencite{beukeleers2022show}.\\

The method of depiction enables individuals to graphically convey the sensory attributes of an object, encompassing its appearance, sound, or texture (Clark, 1996, 2016, 2019; Enfield, 2009; Dingemanse, 2011, 2014, 2015, 2017; Ferrara and Hodge, 2018; Hsu, 2021). Speakers and signers then construct a tangible representation of the object in the immediate present, amalgamating diverse elements representing the objects within the depicted scenario. Depictions are not subject to decryption; instead, they engage the faculty of imagination in the audience, prompting them to envision how the object appears, sounds, or feels (Clark, 1996, 2016, 2019; Enfield, 2009; Dingemanse, 2015; Ferrara and Hodge, 2018). The method of depicting primarily hinges on the utilization of iconic P-signs, such as manual iconic gestures (McNeill, 1992; Kendon, 2004), classifier constructions (Liddell, 2003 for ASL; Johnston and Schembri, 2007, 2010; Ferrara, 2012 for Auslan), and gestural enactment (e.g., Metzger, 1995; Liddell, 2003 for ASL; Cormier et al., 2015 for BSL; McNeill, 1992; Clark, 1996, 2016; Kendon, 2004; Ferrara and Johnston, 2014; Stukenbrock, 2014; Stec et al., 2016 for spoken languages).\\

Expanding Liddell (2003), Dudis (2007) introduces the term "depiction" to elucidate the visually symbolic mapping of semantic elements in signed languages. Dudis posits the existence of additional components within these iconic representations, including the subject (self), vantage point, and temporal progression. Dudis makes a crucial distinction between depicting and non-depicting verbs, emphasizing that depicting verbs vividly portray the events they encapsulate. Signers possess the ability to depict an experiencing self through various means, such as constructed dialogue, constructed action, or handling classifiers. These depictions can either inhabit the life-sized representation or be intricately linked with the self, though not necessarily occupying life-sized space (Stone and Russell, 2016).

Furthermore, Dudis (ibid)  suggests that an event can be depicted without any representation of the self, utilizing techniques for describing an event or concrete object within generic space, event space, real space, or blended space (Stone and Russell, 2016). \\

Depicting verbs can be divided into fundamental conceptual elements, encompassing a figure or a dynamic or stationary entity, background or context against which the figure operates, the specific movement or location, and the course and manner of that movement (Schembri 2003; Taub and Galvan 2001). These components of Depicting Verbs find expression through semantic/entity classifier handshapes, such as "3" representing vehicles in American Sign Language (ASL), or size-and-shape-specific (SASS) handshapes like "F" or "C" to delineate cylindrical objects. Interactions and transitions between these elements are indicated by these handshapes (Brentari, Coppola, Jung, and Goldin-Meadow 2013; Kantor 1980; Schembri 2003; Schick 1990; Taub and Galvan 2001) (Beal-Alvarez, Trussell (2015) \\


 \textcite{beukeleers2022show} stated that signers possess diverse semiotic tools for conveying meaning. Additionally, they challenge the conventional belief that the primary role of a classifier construction, commonly referred to as a depicting sign, is consistently centered on depiction. Their argument proposes that these constructions find their most insightful analysis when examined within the specific context of their use. Signers can employ them for varying degrees of describing, indicating, and depicting meaning.\\


Depicting Verbs aren't subjective; their meanings remain fluid. Unlike static symbols, which can adopt iconic or arbitrary roles, Depicting Verbs maintains an intrinsic connection between their structure and significance. The handshapes within them can symbolize properties of the object or shape they represent or the characteristics of hands as they interact with an entity. Furthermore, the positions and orientations of the hands correspond to the locations and orientations of the things they refer to or the boundaries thereof. At the same time, the gestures mirror real-world motions or the hand's movement when tracing an object \parencite{beuzeville2006visual}.

