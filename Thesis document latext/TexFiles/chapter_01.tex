\chapter{Introduction and Background} \label{sec:introduction}


\section{Background}

This research work is focused on creating an architecture of translation from English text to Irish Sign Language (ISL) based on linguistics analysis and computational linguistic approaches to address the challenges of creating a translation system that is efficient in producing lexical information related to ISL.\\

Nowadays, there is an important advance in technology. However, it is not available to everyone. Excluding some groups, generally impaired people, in this case, deaf and hard hearing people. The exclusion leaves deaf and hard-of-hearing individuals at a disadvantage, aggravating the human-to-human communication barrier while suppressing an already under-resourced set of languages further for the estimated 72 million deaf people in the world (Murtagh, I., Nogales, V. U., \& Blat, J.,2022).\\

Deaf people can communicate with each other using sign language, but if a hearing person has to communicate with a deaf person or vice-versa, communication has many barriers. In such cases, a translation process is required, which can translate the hearing person’s spoken language to the sign language of the deaf person and vice-versa.

\newpage


\section{Motivation}

Many deaf and hard-of-hearing individuals rely on sign
language (SL) on a daily basis as a preferred language (Murtagh, I., Moiselle, R., Leeson, L., 2021).
Although nowadays there are significant advances in spoken
language research, current approaches are often neither
linguistically motivated nor tailored to the unique features of
SLs (De Coster, M., Shterionov, D., Van Herreweghe, M. et al., 2023) . Further research and development are necessary to
Enhance Sign Language Machine Translation (SLMT) and bring
it to a similar level as spoken language MT. This research will
endeavour to improve the accuracy and efficiency of SLMT
systems, making them more accessible to the Deaf community
and empowering deaf and hard-of-hearing individuals to
communicate more effectively with the rest of the world.\\


The lack of information and studies about ISL machine translation is a motivating factor. Machine translation (MT) of verbal language (speech/text) has garnered widespread attention over the past 60 years. On the other hand, the computational processing of signed language has unfortunately not received nearly as much attention, resulting in its exclusion from modern language technologies. Furthermore, MT helps to mitigate the needs of interpreters in Ireland since there is a ratio of 1:45 interpreters for deaf people (Leeson and Venturi, 2017). This impact the communication between hearings and deaf people. 

\newpage




\section{Objectives}

\subsection{General objective}

The general objective of this work is to develop a linguistically motivated sign language machine translation (SLMT) avatar that will translate between English (text) and Irish Sign Language (ISL).



\newpage

\section{Hypothesis}

Our hypothesis is that integration of the theory of grammar Rol and Reference Grammar (RRG) is an adequate approach to lead to an accurate and efficient translation.  

\newpage


\section{Research questions}

\subsection{Question 1}
%What grammar theory is suitable for producing ISL lexicon entries?
How a ISL sentence is constructed ?


\subsection{Question 2}
%What is a SL Classifier, and How Do they present themselves or behave?
How to construct a sentence that is robust enough to capture the linguistic phenomena of classifiers in ISL?

\subsection{Question 3}
%How do we create an SL lexicon entry that is robust enough to capture the linguistic phenomena of classifier predicates?
How do a classifier predicate manifest within a ISL sentence?

\subsection{Question 4}
How can we link the SL lexicon entries to the animation interface when generating a ISL sentence?

\newpage

\section{Methodology}

In this section, we discuss the methodology of this research study. We review the data and resources to use.

\subsection{Description of methodology}

For this project the methodology used is the RRG in the development of our
computational linguistic model for ISL. 

In order to represent the translation, we will create a 3D avatar, considering using the following technologies, MakeHuman or Ready Player Me is an open source technology for create an avatar, Blender or Unity as a rendering engine. 

\newpage

%\section{Organization of the research study}

%\subsection{organization}

\newpage

